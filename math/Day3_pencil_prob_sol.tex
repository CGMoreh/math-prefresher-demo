% Options for packages loaded elsewhere
\PassOptionsToPackage{unicode}{hyperref}
\PassOptionsToPackage{hyphens}{url}
%
\documentclass[
]{article}

\input{pp_preamble.tex}

\begin{document}

\setcounter{section}{3}

\begin{exercise}
\protect\hypertarget{exr:unnamed-chunk-19}{}\label{exr:unnamed-chunk-19}Let \(f(x,y)=x^3 y^4 +e^x -\log y\). What are the following partial derivaitves?

\begin{align*}
\frac{\partial f}{\partial x}(x,y) &=\\
\frac{\partial f}{\partial y}(x,y) &=\\
\frac{\partial^2 f}{\partial x^2}(x,y) &=\\
\frac{\partial^2 f}{\partial x \partial y}(x,y) &= 
\end{align*}
\end{exercise}

\begin{answer}
\noindent
\begin{enumerate}
\item $3x^2y^4 + e^x$
\item $4x^3y^3 - \frac{1}{y}$
\item $6xy^4 + e^x$
\item $12x^2y^3$
\end{enumerate}
\end{answer}

\begin{exercise}[Antiderivative]
\protect\hypertarget{exr:unnamed-chunk-21}{}\label{exr:unnamed-chunk-21}

Find the antiderivative of the following:

\begin{enumerate}
\def\labelenumi{\arabic{enumi}.}
\tightlist
\item
  \(f(x) = \frac{1}{x^2}\)
\item
  \(f(x) = 3e^{3x}\)
\end{enumerate}

\end{exercise}

\begin{answer}
\noindent
\begin{enumerate}
\item $F(x) = -\frac{1}{x} + C$
\item $F(x) = e^{3x} + C$
\end{enumerate}
\end{answer}

\begin{exercise}
\protect\hypertarget{exr:unnamed-chunk-29}{}\label{exr:unnamed-chunk-29}What is the value of \(\int\limits_{-2}^2 e^x e^{e^x} dx\)?
\end{exercise}

\begin{answer}
$e^{e^2} - e^{e^{-2}}$
\end{answer}


\begin{exercise}[Definite integral shortcuts]
\protect\hypertarget{exr:unnamed-chunk-30}{}\label{exr:unnamed-chunk-30}

Simplify the following definite intergrals.

\begin{enumerate}
\def\labelenumi{\arabic{enumi}.}
\tightlist
\item
  \(\int\limits_1^1 3x^2 dx =\)
\item
  \(\int\limits_0^4 (2x+1)dx=\)
\item
  \(\int\limits_{-2}^0 e^x e^{e^x} dx + \int\limits_0^2 e^x e^{e^x} dx =\)
\end{enumerate}

\end{exercise}

\begin{answer}
\noindent
\begin{enumerate}
\item $0$
\item $20$
\item $e^{e^2} - e^{e^{-2}}$
\end{enumerate}
\end{answer}

\begin{example}[Integration by Substitutiton II]
\protect\hypertarget{exm:intsub2}{}\label{exm:intsub2}Simplify \[\int\limits_0^1 \frac{5e^{2x}}{(1+e^{2x})^{1/3}}dx.\]
\end{example}

\begin{answer}
When an expression is raised to a power, it is often helpful to use this expression as the basis for a substitution.  So, let $u=1+e^{2x}$. Then $du=2e^{2x}dx$ and we can set $5e^{2x}dx=5du/2$.    Additionally, $u=2$ when $x=0$ and $u=1+e^2$ when $x=1$.  Substituting all of this in, we get
\begin{align*}
\int\limits_0^1 \frac{5e^{2x}}{(1+e^{2x})^{1/3}}dx
			&= \frac{5}{2}\int\limits_2^{1+e^2}\frac{du}{u^{1/3}}\\
			&= \frac{5}{2}\int\limits_2^{1+e^2} u^{-1/3}du\\
			&= \left. \frac{15}{4} u^{2/3} \right|_2^{1+e^2}\\
			&= 9.53
\end{align*}

\end{answer}

\begin{example}[Integration by Parts]
\protect\hypertarget{exm:unnamed-chunk-31}{}\label{exm:unnamed-chunk-31}Simplify the following integrals. These seemingly obscure forms of integrals come up often when integrating distributions.

\[\int x e^{ax} dx\]
\end{example}

\begin{answer}
Let $u=x$ and $\frac{dv}{dx} = e^{ax}$.  Then $du=dx$ and $v=(1/a)e^{ax}$. Substituting this into the integration by parts formula, we obtain 	
\begin{eqnarray}
\int x e^{ax} dx &=& u v - \int v du\nonumber\\
				&=&x\left( \frac{1}{a}e^{ax}\right) -\int\frac{1}{a}e^{ax}dx\nonumber\\
				&=&\frac{1}{a}xe^{ax}-\frac{1}{a^2}e^{ax}+c\nonumber
\end{eqnarray}

\end{answer}

\end{document}
