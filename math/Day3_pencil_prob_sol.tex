% Options for packages loaded elsewhere
\PassOptionsToPackage{unicode}{hyperref}
\PassOptionsToPackage{hyphens}{url}
%
\documentclass[
]{article}

\usepackage{amsmath,amssymb}
\usepackage{lmodern}
\usepackage{iftex}
\ifPDFTeX
  \usepackage[T1]{fontenc}
  \usepackage[utf8]{inputenc}
  \usepackage{textcomp} % provide euro and other symbols
\else % if luatex or xetex
  \usepackage{unicode-math}
  \defaultfontfeatures{Scale=MatchLowercase}
  \defaultfontfeatures[\rmfamily]{Ligatures=TeX,Scale=1}
\fi
% Use upquote if available, for straight quotes in verbatim environments
\IfFileExists{upquote.sty}{\usepackage{upquote}}{}
\IfFileExists{microtype.sty}{% use microtype if available
  \usepackage[]{microtype}
  \UseMicrotypeSet[protrusion]{basicmath} % disable protrusion for tt fonts
}{}
\makeatletter
\@ifundefined{KOMAClassName}{% if non-KOMA class
  \IfFileExists{parskip.sty}{%
    \usepackage{parskip}
  }{% else
    \setlength{\parindent}{0pt}
    \setlength{\parskip}{6pt plus 2pt minus 1pt}}
}{% if KOMA class
  \KOMAoptions{parskip=half}}
\makeatother
\usepackage{xcolor}
\usepackage[margin=1.5in]{geometry}
\usepackage{color}
\usepackage{fancyvrb}
\newcommand{\VerbBar}{|}
\newcommand{\VERB}{\Verb[commandchars=\\\{\}]}
\DefineVerbatimEnvironment{Highlighting}{Verbatim}{commandchars=\\\{\}}
% Add ',fontsize=\small' for more characters per line
\usepackage{framed}
\definecolor{shadecolor}{RGB}{248,248,248}
\newenvironment{Shaded}{\begin{snugshade}}{\end{snugshade}}
\newcommand{\AlertTok}[1]{\textcolor[rgb]{0.94,0.16,0.16}{#1}}
\newcommand{\AnnotationTok}[1]{\textcolor[rgb]{0.56,0.35,0.01}{\textbf{\textit{#1}}}}
\newcommand{\AttributeTok}[1]{\textcolor[rgb]{0.77,0.63,0.00}{#1}}
\newcommand{\BaseNTok}[1]{\textcolor[rgb]{0.00,0.00,0.81}{#1}}
\newcommand{\BuiltInTok}[1]{#1}
\newcommand{\CharTok}[1]{\textcolor[rgb]{0.31,0.60,0.02}{#1}}
\newcommand{\CommentTok}[1]{\textcolor[rgb]{0.56,0.35,0.01}{\textit{#1}}}
\newcommand{\CommentVarTok}[1]{\textcolor[rgb]{0.56,0.35,0.01}{\textbf{\textit{#1}}}}
\newcommand{\ConstantTok}[1]{\textcolor[rgb]{0.00,0.00,0.00}{#1}}
\newcommand{\ControlFlowTok}[1]{\textcolor[rgb]{0.13,0.29,0.53}{\textbf{#1}}}
\newcommand{\DataTypeTok}[1]{\textcolor[rgb]{0.13,0.29,0.53}{#1}}
\newcommand{\DecValTok}[1]{\textcolor[rgb]{0.00,0.00,0.81}{#1}}
\newcommand{\DocumentationTok}[1]{\textcolor[rgb]{0.56,0.35,0.01}{\textbf{\textit{#1}}}}
\newcommand{\ErrorTok}[1]{\textcolor[rgb]{0.64,0.00,0.00}{\textbf{#1}}}
\newcommand{\ExtensionTok}[1]{#1}
\newcommand{\FloatTok}[1]{\textcolor[rgb]{0.00,0.00,0.81}{#1}}
\newcommand{\FunctionTok}[1]{\textcolor[rgb]{0.00,0.00,0.00}{#1}}
\newcommand{\ImportTok}[1]{#1}
\newcommand{\InformationTok}[1]{\textcolor[rgb]{0.56,0.35,0.01}{\textbf{\textit{#1}}}}
\newcommand{\KeywordTok}[1]{\textcolor[rgb]{0.13,0.29,0.53}{\textbf{#1}}}
\newcommand{\NormalTok}[1]{#1}
\newcommand{\OperatorTok}[1]{\textcolor[rgb]{0.81,0.36,0.00}{\textbf{#1}}}
\newcommand{\OtherTok}[1]{\textcolor[rgb]{0.56,0.35,0.01}{#1}}
\newcommand{\PreprocessorTok}[1]{\textcolor[rgb]{0.56,0.35,0.01}{\textit{#1}}}
\newcommand{\RegionMarkerTok}[1]{#1}
\newcommand{\SpecialCharTok}[1]{\textcolor[rgb]{0.00,0.00,0.00}{#1}}
\newcommand{\SpecialStringTok}[1]{\textcolor[rgb]{0.31,0.60,0.02}{#1}}
\newcommand{\StringTok}[1]{\textcolor[rgb]{0.31,0.60,0.02}{#1}}
\newcommand{\VariableTok}[1]{\textcolor[rgb]{0.00,0.00,0.00}{#1}}
\newcommand{\VerbatimStringTok}[1]{\textcolor[rgb]{0.31,0.60,0.02}{#1}}
\newcommand{\WarningTok}[1]{\textcolor[rgb]{0.56,0.35,0.01}{\textbf{\textit{#1}}}}
\usepackage{longtable,booktabs,array}
\usepackage{calc} % for calculating minipage widths
% Correct order of tables after \paragraph or \subparagraph
\usepackage{etoolbox}
\makeatletter
\patchcmd\longtable{\par}{\if@noskipsec\mbox{}\fi\par}{}{}
\makeatother
% Allow footnotes in longtable head/foot
\IfFileExists{footnotehyper.sty}{\usepackage{footnotehyper}}{\usepackage{footnote}}
\makesavenoteenv{longtable}
\usepackage{graphicx}
\makeatletter
\def\maxwidth{\ifdim\Gin@nat@width>\linewidth\linewidth\else\Gin@nat@width\fi}
\def\maxheight{\ifdim\Gin@nat@height>\textheight\textheight\else\Gin@nat@height\fi}
\makeatother
% Scale images if necessary, so that they will not overflow the page
% margins by default, and it is still possible to overwrite the defaults
% using explicit options in \includegraphics[width, height, ...]{}
\setkeys{Gin}{width=\maxwidth,height=\maxheight,keepaspectratio}
% Set default figure placement to htbp
\makeatletter
\def\fps@figure{htbp}
\makeatother
\setlength{\emergencystretch}{3em} % prevent overfull lines
\providecommand{\tightlist}{%
  \setlength{\itemsep}{0pt}\setlength{\parskip}{0pt}}
\setcounter{secnumdepth}{5}
\usepackage{booktabs}

\usepackage{epsfig}
\usepackage{epstopdf}
\usepackage{rotate}
\usepackage{graphicx}
\usepackage{hyperref}
\usepackage{alphalph}
\usepackage{caption}
\usepackage[hang,flushmargin]{footmisc}
\usepackage{framed}
\usepackage{xcolor}
\usepackage{verbatim} 

\usepackage{bm}
\setcounter{MaxMatrixCols}{20}
\newcommand{\Var}{\mathrm{Var}}
\newcommand{\SD}{\mathrm{SD}}
\newcommand{\Cov}{\mathrm{Cov}}
\newcommand{\fx}{f({\bf x})}
\newcommand\R{{\textsf R~}}
\newcommand\Rst{\textsf{RStudio}}

% spacing between environments
\usepackage{amsthm}
\makeatletter
\def\thm@space@setup{%
  \thm@preskip=15pt plus 2pt minus 4pt
  \thm@postskip=\thm@preskip
}
\makeatother

% link colors in pdf?
\usepackage{xcolor}  
\definecolor{crimson}{RGB}{204,0,0}
\hypersetup{
  colorlinks = true,
  urlcolor = crimson,
  linkbordercolor = {white},
  linkcolor = crimson
}


% Title format
\usepackage{titling}
\pretitle{\Huge\sffamily}
\posttitle{\par\vskip 1em}
\predate{\LARGE\sffamily}
\postdate{\par}

\urlstyle{tt}
\ifLuaTeX
  \usepackage{selnolig}  % disable illegal ligatures
\fi
\usepackage[]{natbib}
\bibliographystyle{apalike}
\IfFileExists{bookmark.sty}{\usepackage{bookmark}}{\usepackage{hyperref}}
\IfFileExists{xurl.sty}{\usepackage{xurl}}{} % add URL line breaks if available
\urlstyle{same} % disable monospaced font for URLs
\hypersetup{
  pdftitle={Math Prefresher for Political Scientists},
  hidelinks,
  pdfcreator={LaTeX via pandoc}}

\title{Math Prefresher for Political Scientists}
\author{}
\date{\vspace{-2.5em}August 2023}

\usepackage{amsthm}
\newtheorem{theorem}{Theorem}[section]
\newtheorem{lemma}{Lemma}[section]
\newtheorem{corollary}{Corollary}[section]
\newtheorem{proposition}{Proposition}[section]
\newtheorem{conjecture}{Conjecture}[section]
\theoremstyle{definition}
\newtheorem{definition}{Definition}[section]
\theoremstyle{definition}
\newtheorem{example}{Example}[section]
\theoremstyle{definition}
\newtheorem{exercise}{Pencil Problem}[section]
\theoremstyle{definition}
\newtheorem{hypothesis}{Hypothesis}[section]
\theoremstyle{remark}
\newtheorem*{remark}{Remark}
\newtheorem*{solution}{Solution}
\newtheorem*{answer}{Answer}

\DeclareUnicodeCharacter{2212}{-}


\begin{document}

\setcounter{section}{3}

\begin{exercise}
\protect\hypertarget{exr:unnamed-chunk-19}{}\label{exr:unnamed-chunk-19}Let \(f(x,y)=x^3 y^4 +e^x -\log y\). What are the following partial derivaitves?

\begin{align*}
\frac{\partial f}{\partial x}(x,y) &=\\
\frac{\partial f}{\partial y}(x,y) &=\\
\frac{\partial^2 f}{\partial x^2}(x,y) &=\\
\frac{\partial^2 f}{\partial x \partial y}(x,y) &= 
\end{align*}
\end{exercise}

\begin{answer}
\noindent
\begin{enumerate}
\item $3x^2y^4 + e^x$
\item $4x^3y^3 - \frac{1}{y}$
\item $6xy^4 + e^x$
\item $12x^2y^3$
\end{enumerate}
\end{answer}

\begin{exercise}[Antiderivative]
\protect\hypertarget{exr:unnamed-chunk-21}{}\label{exr:unnamed-chunk-21}

Find the antiderivative of the following:

\begin{enumerate}
\def\labelenumi{\arabic{enumi}.}
\tightlist
\item
  \(f(x) = \frac{1}{x^2}\)
\item
  \(f(x) = 3e^{3x}\)
\end{enumerate}

\end{exercise}

\begin{answer}
\noindent
\begin{enumerate}
\item $F(x) = -\frac{1}{x} + C$
\item $F(x) = e^{3x} + C$
\end{enumerate}
\end{answer}

\begin{exercise}
\protect\hypertarget{exr:unnamed-chunk-29}{}\label{exr:unnamed-chunk-29}What is the value of \(\int\limits_{-2}^2 e^x e^{e^x} dx\)?
\end{exercise}

\begin{answer}
$e^{e^2} - e^{e^{-2}}$
\end{answer}


\begin{exercise}[Definite integral shortcuts]
\protect\hypertarget{exr:unnamed-chunk-30}{}\label{exr:unnamed-chunk-30}

Simplify the following definite intergrals.

\begin{enumerate}
\def\labelenumi{\arabic{enumi}.}
\tightlist
\item
  \(\int\limits_1^1 3x^2 dx =\)
\item
  \(\int\limits_0^4 (2x+1)dx=\)
\item
  \(\int\limits_{-2}^0 e^x e^{e^x} dx + \int\limits_0^2 e^x e^{e^x} dx =\)
\end{enumerate}

\end{exercise}

\begin{answer}
\noindent
\begin{enumerate}
\item $0$
\item $20$
\item $e^{e^2} - e^{e^{-2}}$
\end{enumerate}
\end{answer}

\begin{example}[Integration by Substitutiton II]
\protect\hypertarget{exm:intsub2}{}\label{exm:intsub2}Simplify \[\int\limits_0^1 \frac{5e^{2x}}{(1+e^{2x})^{1/3}}dx.\]
\end{example}

\begin{answer}
When an expression is raised to a power, it is often helpful to use this expression as the basis for a substitution.  So, let $u=1+e^{2x}$. Then $du=2e^{2x}dx$ and we can set $5e^{2x}dx=5du/2$.    Additionally, $u=2$ when $x=0$ and $u=1+e^2$ when $x=1$.  Substituting all of this in, we get
\begin{align*}
\int\limits_0^1 \frac{5e^{2x}}{(1+e^{2x})^{1/3}}dx
			&= \frac{5}{2}\int\limits_2^{1+e^2}\frac{du}{u^{1/3}}\\
			&= \frac{5}{2}\int\limits_2^{1+e^2} u^{-1/3}du\\
			&= \left. \frac{15}{4} u^{2/3} \right|_2^{1+e^2}\\
			&= 9.53
\end{align*}

\end{answer}

\begin{example}[Integration by Parts]
\protect\hypertarget{exm:unnamed-chunk-31}{}\label{exm:unnamed-chunk-31}Simplify the following integrals. These seemingly obscure forms of integrals come up often when integrating distributions.

\[\int x e^{ax} dx\]
\end{example}

\begin{answer}
Let $u=x$ and $\frac{dv}{dx} = e^{ax}$.  Then $du=dx$ and $v=(1/a)e^{ax}$. Substituting this into the integration by parts formula, we obtain 	
\begin{eqnarray}
\int x e^{ax} dx &=& u v - \int v du\nonumber\\
				&=&x\left( \frac{1}{a}e^{ax}\right) -\int\frac{1}{a}e^{ax}dx\nonumber\\
				&=&\frac{1}{a}xe^{ax}-\frac{1}{a^2}e^{ax}+c\nonumber
\end{eqnarray}

\end{answer}

\end{document}
