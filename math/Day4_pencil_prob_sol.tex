% Options for packages loaded elsewhere
\PassOptionsToPackage{unicode}{hyperref}
\PassOptionsToPackage{hyphens}{url}
%
\documentclass[
]{article}

\input{pp_preamble.tex}

\begin{document}

\setcounter{section}{4}

\begin{exercise}[Probability]
\protect\hypertarget{exr:prob1}{}\label{exr:prob1}

Suppose you had a pair of four-sided dice. You sum the results from a single toss. Let us call this sum, or the outcome, X.

\begin{enumerate}
\def\labelenumi{\arabic{enumi}.}
\item
  What is \(P(X = 5)\), \(P(X = 3)\), \(P(X = 6)\)?
\item
  What is \(P(X=5 \cup X = 3)^C\)?
\end{enumerate}

\end{exercise}

\begin{answer}
\noindent
\begin{enumerate}
\item $P(X = 5) = \frac{4}{16}$, $P(X = 3) = \frac{2}{16}$, $P(X = 6) = \frac{3}{16}$
\item $P(X=5 \cup X = 3)^C = \frac{10}{16}$
\end{enumerate}
\end{answer}

\begin{example}[Bayes' Rule]
\protect\hypertarget{exm:bayesrule}{}\label{exm:bayesrule}In a given town, 40\% of the voters are Democrat and 60\% are Republican. The president's budget is supported by 50\% of the Democrats and 90\% of the Republicans. If a randomly (equally likely) selected voter is found to support the president's budget, what is the probability that they are a Democrat?
\end{example}

\begin{answer}
We are given that
$$P(D) = .4, P(D^c) = .6, P(S|D) = .5, P(S|D^c) = .9$$
Using this, Bayes' Law and the Law of Total Probability, we know: 

$$P(D|S) = \frac{P(D)P(S|D)}{P(D)P(S|D) + P(D^c)P(S|D^c)}$$
$$P(D|S) = \frac{.4 \times .5}{.4 \times .5 + .6 \times .9 } = .27$$
\end{answer}

\begin{exercise}[Expectation and Variance]
\protect\hypertarget{exr:expvar}{}\label{exr:expvar}

Suppose we have a PMF with the following characteristics:
\begin{eqnarray*}
  P(X = -2) = \frac{1}{5}\\
  P(X = -1) = \frac{1}{6}\\
  P(X = 0) = \frac{1}{5}\\
  P(X = 1) = \frac{1}{15}\\
  P(X = 2) = \frac{11}{30}
\end{eqnarray*}

\begin{enumerate}
\def\labelenumi{\arabic{enumi}.}
\tightlist
\item
  Calculate the expected value of X
\end{enumerate}

Define the random variable \(Y = X^2\).

\begin{enumerate}
\def\labelenumi{\arabic{enumi}.}
\setcounter{enumi}{1}
\item
  Calculate the expected value of Y. (Hint: It would help to derive the PMF of Y first in order to calculate the expected value of Y in a straightforward way)
\item
  Calculate the variance of X.
\end{enumerate}

\end{exercise}

\begin{answer}
\noindent
\begin{enumerate}
\item $E(X) = -2(\frac{1}{5}) + -1(\frac{1}{6}) + 0(\frac{1}{5}) + 1(\frac{1}{15}) + 2(\frac{11}{30}) = \frac{7}{30}$
\item $E(Y) = 0(\frac{1}{5}) + 1(\frac{7}{30}) + 4(\frac{17}{30}) = \frac{5}{2}$
\item \begin{align*}
\text{Var}(X) &= E[X^2] - E[X]^2\\
&= E(Y) - E(X)^2\\
&= \frac{5}{2} - \frac{7}{30}^2 \approx 2.45
\end{align*}
\end{enumerate}
\end{answer}


\end{document}
